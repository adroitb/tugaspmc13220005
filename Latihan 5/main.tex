\documentclass[conference]{IEEEtran}
\usepackage{algorithm2e}
\usepackage{cite}
\usepackage{graphicx}
\usepackage[mathletters]{ucs}
\usepackage[utf8x]{inputenc}


% Judul
\title{Implementasi Algoritma Dijkstra Dalam
Menemukan Jarak Terdekat Dari Lokasi Pengguna
Ke Tanaman Yang Di Tuju}

\author{\IEEEauthorblockN{Adro Anra Purnama}
\IEEEauthorblockA{\textit{School of Electrical Engineering and Informatics}\\
\textit{Institut Teknologi Bandung}\\
Bandung, Indonesia\\
Email: 13220005@std.stei.itb.ac.id}
}

%folder gambar
\graphicspath{{./gambar/}}
\begin{document}

\maketitle
\begin{abstract}
    Kebun Raya Purwodadi dengan luas area sekitar 85
hektar ternyata kekurangan papan informasi yang menyebabkan
pengunjung kerap kali kebingungan dalam mencari lokasi tana-
man tertentu. Paper ini bertujuan untuk membuat simulasi
dari algoritma yang dapat menentukan jarak terdekat antara
pengunjung (pengguna program) dengan lokasi tanaman yang
dituju. Algoritma yang digunakan adalah algoritma Dijkstra
yang beroperasi secara menyeluruh (greedy) untuk menguji
seitap persimpangan (Vertex) dan jalan (Edge) pada Kebun
Raya Purwodadi. Berdasarkan hasil simulasi dan pengujian,
kompleksitas ruang dari program ini adalah O(V) karena adanya
pembentukan array yang berisi V nodes untuk mencari heap min-
imum. Sementara, kompleksitas waktu dari algoritma tersebut
adalah O(V2 ).
\end{abstract}   

\begin{IEEEkeywords}
    Dijkstra, Vertex, Edge, Tanaman.
\end{IEEEkeywords}

\section{Introduction}
Studi mengenai penggunaan algoritma Dijkstra dalam men-cari jarak terdekat dapat diimplementasikan pada kasus pen-carian tanaman pada Kebun Raya Purwodadi seperti yang telah
dilakukan oleh Yusuf et al di tahun 2017 [1]. Paper ini bertu-juan untuk melakukan simulasi kembali terhadap penelitian
yang telah dilakukan dengan bahasa C serta mengevaluasi
efisiensinya melalui perhitungan kompleksitas waktu dan ru-ang dengan analisis Big-O.

Di Kecamatan Purwodadi, Kabupaten Pasuruan, terdapat
salah satu kebun raya di Indonesia yang bernama Kebun
Raya Purwodadi yang memiliki luas area hingga 85 hektar.
Kebun raya sebagai fasilitas rekreasi dan penelitian ini ternyata
kekurangan papan informasi yang seharusnya disediakan oleh
pihak pengelola. Hal ini menyebabkan banyaknya pengunjung
yang merasa kebingungan untuk mencari lokasi dari tanaman
tertentu. Oleh karena itu, Yusuf et al (2017) memutuskan
untuk membuat suatu aplikasi dengan memanfaatkan algoritma
Dijkstra untuk membantu pengunjung Kebun Raya Purwodadi
dalam mencari lokasi tertentu.

Algoritma Dijkstra digunakan karena algoritma ini dapat
beroperasi secara menyeluruh (algoritma greedy) terhadap
semua alternatif fungsi serta durasi eksekusi yang lebih cepat
jika dibandingkan dengan algoritma serupa, yaitu Bellman-Ford. Algoritma ini akan mencari jalur dengan ’biaya’ atau
cost terendah antara dua titik dengan membandingkan semua
alternatif yang ada.

Pada kasus ini, masing-masing persimpangan di Kebun
Raya Purwodadi direpresentasikan sebagai vertex dan setiap
jalan direpresentasikan sebagai edge. Rute terdekat yang dida-patkan akan diperoleh dari pembobotan setiap vertex dan edge
berdasarkan jarak antara titik pengguna dengan titik tujuan
atau tanaman.

\section{Studi Pustaka}
A. \textit{Algoritma Dijkstra}

\SetKwComment{Comment}{/* }{ */}
\RestyleAlgo{ruled}
\begin{algorithm}
    \caption{Dijkstra’s Algorithm \textit{Dijkstra}}
    \KwResult{Find the shortest path from a to z}
    \SetAlgoLined\DontPrintSemicolon
    \SetKwFunction{proc}{proc}
    \SetKwProg{myproc}{procedure}{}{}
    \myproc{\textit{Dijkstra} (G: weighted connected simple
    graph, with all weights positive)
    }{}
    {$G$ has vertices $a = v_0, v_1 , .... , v_n$ = z and lenghts 
    $w(v_i , v_J)$ where $w(v_i , v_j) = \inf$ if $v_i , v_j$ is not
    an edge in $G$}

    \If{i := 1 to n}{
        $L(v_i) := \infty$
    }
    $L(a) := 0$

    $S :=   ∅$

    {the labels are now initialized so that the label of a is
    0 and all other labels are ∞, and S is the empty set}
    
    \While{z ∈/ S}{
        $u :=$ a vertex not in $S$ with $L(u)$ minimal

        $S := S$  ∪ (u)
        \For{\textit{all vertices v not in} $S$}{
            \If{$L(u) + w(u,v) < L(v)$}{
                $L(u) + w(u,v) < L(v)$

                {this adds a vertex to $S$ with minimal label 
                and updates the labels of vertices not in $S$}
            }
        }
    }
    \KwRet{$L(z)$ \textit{length of a shortest path from a to z}}
\end{algorithm}

Algoritma Dijkstra adalah algoritma yang digunakan untuk
menemukan jarak jalur terpendek antara dua vertice pada
graph berbobot dan tidak berarah sederhana [2]. Berbobot
berarti grafik memiliki edge dengan suatu 'bobot' atau harga.
Bobot dapat merepresentasikan jarak, waktu, atau apapun
yang memodelkan koneksi antara kedua node. Tidak berarah
memiliki arti bahwa untuk setiap node yang terhubung, kita
dapat mendekati suatu node dari kedua arah. Pendekatan Di-jikstra juga memiliki asumsi bahwa bobot pada edge memiliki
nilai yang tidak negatif. Hal ini karena nilai bobot akan
terus dibandingkan dan diambil nilai yang paling kecil. Ada
banyak varian pada algoritma ini, namun pada percobaan
ini digunakan varian dimana suatu node ditetapkan menjadi
source node. Dari node inilah akan dicari jarak terpendek
diantara node lain. Algoritma ini dicetuskan oleh Edsger
Wybe Dijkstra, salah seorang tokoh ternama di bidang com-puter science [3]. Kompleksitas dari algoritma dijkstra adalah
\textit{O(n$^2$)}, dengan n menyatakan jumlah vertice dari graph yang
bersangkutan

B. \textit{Kebun Raya Purwodadi}

Kebun Raya Purwodadi adalah kebun penelitian di Keca-matan Purwodadi, Jawa Timur. Ia juga dikenal dengan nama
Hortus Ilkim Kering Purwodadi dan didirikan tanggal 30 Jan-uari 1941 oleh Dr. L.G.M. Baas Becking. Sebagai cabang dari
Kebun Raya Bogor, ia memiliki fungsi mengkoleksi tumbuhan
yang hidup di dataran rendah kering. Sebagai Balai Konservasi
Tumbuhan di bawah Pusat Konservasi Tumbuhan Kebun Raya,
Kedeputian Bidang Ilmu Pengetahuan Hayati LIPI, kebun raya
ini memiliki banyak tumbuhan yang dinaunginya. Dengan
menggunakan algoritma Dijkstra, diharapkan ia dapat mem-bantu pengunjung mencari tanaman tertentu maupun jarak
yang paling optimal.


\section{Metodologi Penelitian}
Peneliti menggunakan beberapa tahap dalam penyusunan
paper ini. Pertama, dilakukan pengkajian dan studi literatur
dengan membaca referensi paper yang berkaitan dan memilih
paper yang dapat menjadi acuan dalam penelitian yang di-lakukan, sehingga dari pilihan topik dan tema yang berkaitan
secara luas dapat dikecilkan menjadi sebuah paper yang men-cakup mayoritas dari topik yang dibahas. Setelah ditemukan
beberapa paper, dilakukan perangkuman untuk menentukan
paper yang sesuai sekaligus membahas poin-poin penting
dari paper yang ingin dicapai. Setelah kedua tahap tersebut
dilewati, penentuan paper yang dijadikan prototype penelitian
merupakan hal yang mudah dan menjadi titik pencapaian
dalam studi literatur dan pemilihan topik dari prototype peneli-tian yang dilakukan.

Setelah itu, tahap selanjutnya yang dilakukan oleh peneliti
adalah pembuatan prototype berupa program yang ditulis
dalam bahasa C. Pembuatan prototype berupa kode ini di-lakukan terus-menerus dengan menggunakan metode trial and
error sehingga perlu dilakukan revisi hingga protoype kode
yang dibuat dapat mendapatkan output yang optimal dan
sesuai dengan spesifikasi yang diharapakan. Tahap terakhir
enelitian adalah pemaparan kode yang berhasil di-jalankan tersebut ke dalam paper.

\begin{figure}[htbp]
    \scalebox{0.9}{\input{{gambar/alur.pdf_tex}}}
    \centering
\end{figure}

\section{Implementasi Dan Pengujian}
A. \textit{Implementasi Graph pada Array dalam Bahasa C}

Program akan dimulai dengan pembacaan file bernama
listtanaman.txt. File tersebut akan menyimpan informasi men-genai semua nama tanaman yang bersangkutan. Setelah pem-bacaan tersebut, akan dicari informasi mengenai bobot graph
yang menghubungkan node. Informasi ini disimpan di dalam
matriks segitiga bawah kiri didalam file \textit{jarakantarpohon.txt}
yang juga dibuka saat program dijalankan. Matriks menggam-barkan bobot antara jarak dua node tanaman sekali saja karena
pemodelan \textit{undirected graph} yang memiliki jarak sama baik
dari a ke b maupun b ke a. Nilai -1 akan menggambarkan
bagian node yang tidak terhubung sama sekali dalam graph
dan juga dinyatakan dalam suatu variabel bernama int max
(Jaraknya sebesar tak hingga). Nilai jarak terpendek akan
disimpan dalam array tersebut selagi program berjalan.
\vspace{5mm} %5mm vertical space

B. \textit{Implementasi Algoritma Dijkstra dalam Bahasa C}

Dalam implementasi algoritma, abstraksi dengan menggu-nakan pseudocode dapat dibagi menjadi dua buah fungsi dan
satu program utama. Fungsi yang digunakan adalah fungsi
printgraph (Fungsi Graph) untuk memunculkan graph beruku-ran n x n ke layar pengguna. Algoritma dari fungsi tersebut
dapat dilihat pada bagian di bawah ini:

\begin{algorithm}
    \caption{Fungsi Graph}
    \KwResult{Memunculkan Graph $n x n$ Ke Layar}
    \SetKwProg{myproc}{procedure}{}{}
    \myproc{printgraph(n, graph[n][n])}{}{}
    \While{$  i ≤ n − 1 $}{
        $ j  ← 0;$
        \While{$ j ≤ n − 1$}{
            \uIf{$graph[i][j] = int_max$}{
                \textbf{output}(-1);
            }
            \Else(){
                \textbf{output}($graph[i][j]$);
            }
            
            $j ← j + 1$;
        }

        $i ← i + 1;$
    }
\end{algorithm}

Fungsi kedua yang digunakan adalah fungsi pencari indeks
pada array yang akan diproses dengan menggunakan pendekatan algoritma Dijkstra. Abstraksi fungsi yang digunakan
dapat dilihat pada bagian berikut ini:

\begin{algorithm}
    \caption{Fungsi Pencari indeks}
    \KwResult{Mencari indeks yang akan diproses dengan algoritma Dijkstra}
    \textbf{Initialization:}
    
    $is\_found ← FALSE$;

    $i ← 0$;

    \textbf{Algorithm:}

    \While{$i ≤ n − 1$}{
        $j ← 0;$
        \If{$!is\_final[i]$  \textbf{and}  $!is\_found$}{
            $idx\_min ← i;$

            $val\_minimum ← jarak\_f[i]$

            $is\_found ← true;$

        }
        \If{$is\_final[i]$  \textbf{and}  $!is\_found$ \textbf{and}
        ($jarak\_f[i] < val\_minimum$)}{
            $idx_min ← i;$

            $val\_minimum ← jarak\_f[i];$
        }
    }
    \uIf{$is\_found$}{
        \textbf{return}($idx\_min$)
    }
    \Else{
        \textbf{return}($int\_max$)
    }
\end{algorithm}

Program utama akan membaca file database tanaman
beserta jarak masing-masing tanaman dan akan mencetak
daftar tanaman yang berada di Kebun Raya Purwodadi.
Kemudian, program akan menerima input salah satu tanaman
terdekat dari pengguna sebagai penanda posisi awal pengguna.
Setelah itu, program akan kembali menerima input posisi
tanaman tujuan dan memproses pencarian rute terdekat dengan
algoritma Dijkstra. Rute yang diperlukan akan ditampilkan
dalam bentuk list nama tanaman yang harus dilalui pengguna
dan menampilkan jarak antara kedua tanaman tersebut.
Implementasi algoritma dalam abstraksi tersebut dapat dilihat
pada gambar di bawah ini:

\begin{algorithm}
    \caption{Program Utama Pencarian Rute Antara Dua Tanaman -
    Pembacaan Jumlah Tanaman}
    \KwResult{Menyimpan nama tanaman pada sebuah array}
    \textbf{Algorithm:}

    \textbf{input}(namafile\_tanaman);

    \textbf{open}(namafile\_tanaman);

    \textbf{read}(namafile\_tanaman);
    
    n\_tanaman ← 0;

    baris ← 0;

    \While{$baris ≤ max\_len $}{
        token ← parse(baris);

        token ← nama\_tanaman[n\_tanaman];

        n\_tanaman ← n\_tanaman + 1;

        baris ← baris+1;
    }
\end{algorithm}

Setelah pembacaan jumlah tanaman dari file, maka diper-lukan graph atau jarak antar tanaman yang akan menjadi dasar
perhitungan dari pencarian rute terdekat. Proses memasukkan
graph dapat dilihat pada algoritma berikut ini:

\begin{algorithm}
    \caption{Program Utama Pencarian Rute Antara Dua Tanaman -
    Memasukkan Graph}
    \KwResult{Menyimpan graph dalam sebuah matriks $n$×$n$}

    \textbf{input}(namafile\_graph);

    \textbf{open}(namafile\_graph);

    \textbf{read}(namafile\_graph);

    baris ← 0;

    \While{baris≤ max\_len}{
        $k ← 0;$

        $token ← parse(baris)$

        \While{token != NULL}{
            graph[$j$]$[k$] ← $token$;

            graph[$k$]$j$] ← $token$;

            \uIf{token == -1}{
                graph[$j$]$[k$] ← $int\_max$;

                graph[$k$]$j$] ← $int\_max$;
            }
            \Else{
                $k ← k + 1;$

                $token ← parse(NULL)$
            }
        }
        $baris ← baris + 1;$
    }
\end{algorithm}

Setelah data yang dibutuhkan dimasukkan, implementasi
dari algoritma Dijkstra untuk pencarian rute terdekat adalah
sebagai berikut:

\begin{algorithm}
    \caption{Program Utama Pencarian Rute Antara Dua Tanaman: Pencarian
    Jarak dengan Algoritma Di-jkstra}
    \textbf{Algorithm:}

    \textbf{input}(idx\_a);

    idx\_a ← idx\_a-1;

    \textbf{input}(idx\_tujuan);

    idx\_tujuan ← idx\_tujuan - 1;

    \For{$i = 0$ \textbf{to} n\_tanaman}{
        \uIf{i = idx\_a}{
            jarak\_f[i] ← 0;

            is\_final[i] ← $FALSE$;
        }
        \Else(){
            jarak\_f[i] ← int\_max;

            is\_final[i] ← $FALSE$;
        }
        \For{j = 0 \textbf{to} n\_tanaman}{
            list\_dilalui[i][j] ← int\_max;
        }
        idx\_lalui[i] ← 0;
    }

    jarak\_f[idx\_a] ← 0;

    list\_dilalui[idx\_a][0] ← idx\_a;

    idx\_lalui[idx\_a] ← idx\_lalui[idx\_a]+1;

    idx\_now ← idx\_a;
    
    \While{idx\_now != int\_max}{
        is\_final[idx\_now] ← $TRUE$;

        \For{i = 0 \textbf{to} n\_tanaman-1}{
            \If{(!is\_final[i]) \textbf{and} graph[idx\_now][i] !=
             int\_max \textbf{and} (jarak\_f[idx\_now][i] > jarak\_f[i])}{
                jarak\_f[i] ← (jarak\_f[idx\_now][i]);
                idx\_lalui ← idx\_lalui[idx\_now] +1;
             }
        \For{j = 0 \textbf{to} idx\_dilalui[i] - 1}{
            \uIf{j = idx\_dilalui[i] - 1}{
                list\_dilalui[i][j] ← i;
            }
            \Else(){
                list\_dilalui[i][j] ← list\_dilalui[idx\_now][j];
            }
        }
        }
        idx\_now ← idx\_process(n\_tanaman, jarak\_f, is\_final);
    }
\end{algorithm}

C. \textit{Implementasi Program dalam Bahasa C}

Implementasi program dalam bahasa C dapat dilihat
pada repository berikut. https://github.com/ReynaldoAverill/
Tugas7PMC


D. \textit{Perhitungan Kompleksitas Waktu}

Kompleksitas dari program ini dengan notasi kompleksitas
Big O adalah $O(n^2)$. Hal tersebut disebabkan pada loop
program bagian for, terdapat loop for lain yang berjumlah
dua loop (Terletak pada bagian assign kondisi awal dan ketika
program menjalankan algoritma Djikstra). Karena hal tersebut,
akibatnya adalah kompleksitas waktu akan naik seiring dengan
naiknya n program yang dijalankan, namun tidak bersifat
linear sehingga kompleksitas waktunya adalah $O(n^2)$. Grafik
kompleksitas waktu dapat direpresentasikan pada gambar 1.



\bibliographystyle{IEEEtran}
\bibliography{reference.bib}

\end{document}